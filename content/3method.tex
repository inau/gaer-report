\section{Evolving chairs using NEAT}
To solve a problem using genetic algorithms, we need an encoding
consisting of a genotype, phenotype, and a mapping from genotype to phenotype.

Encodings can generally be divided into two categories; direct and generative
(or indirect). Generative encodings have several properties that lend themselves
well to our problem domain. By using a genotype encoding that \emph{generates}
the phenotype, a single sub-structure in the genome can encode several similar
sub-structures in the phenome. This makes it significantly easier to evolve
individuals that have symmetry or repetition, but with variation --- such as
chairs! While the Hornby study\cite{paper:ev4} uses Lindenmayer
systems\cite{Hornby2003}, we thought it would be interesting to try a
neuroevolutionary approach.

\subsection{Genetic Encoding}
As suggested by Stanley\cite{Stanley2007}, and as Clune and
Lipson\cite{Clune:2011:EOG:2078245.2078246} does, we consider the evolved CPPNs
to be a description of our phenotype, and query them with 3 Cartesian
coordinates, representing a point in 3d space. The output of the network then
determines the absence or presence of a voxel in that point, and hence whether
that point will be part of the chair's design. A voxel is present in a point if
the network outputs a vale larger than or equal to $0.5$ for that point.

This gives us a binary utility matrix representing a chair. To be able to render
and simulate physical experiments on it, we use the grassfire tranform, first
introduced by Harry Blum\cite{blum67}, to find a single connected component (If
there is more than one, we take the largest and discard the rest). We then
"hollow out" the chair --- we remove all voxels that are completely surrounded by
voxels. This is primarily to avoid representing them in the physical simulation,
which is computationally expensive.

\subsection{Evaluation}
The evaluation(fitness) function determines the value of each individual, in
each generation of the evolutionary experiment. For our fitness function we
considered the use of several static analysis terms, physical experiments, and
an interactive term.

For the static term, we considered analysis of sitting-surface area and angle,
but ended up relying on the physical simulation for this definition of
"chairdness" instead. The only static term we used in the end is the number of
voxels a chair consists of, corresponding to a measure of material cost.

for the physical simulation, we wanted to measure stability, as well as
viability as an object on which people can sit. In initial phases of the
project, we performed the experiment by dropping a sphere on the chair, and then
measuring the resulting rotation of the chair, the horizontal movement of the
sphere, and the height in which the sphere came to rest. In later stages, we
came to perform a similar experiment, but with a human rag doll. The rag doll
gave us the same information as the sphere, but with added opportunity to
measure a term for posture, by taking the vertical distance between the head and
hips.

The interactive term is used only conditionally. After evolving generations
without interaction (we have experimented with one, tens, and hundredths of
generations between interaction), the user is allowed (but not forced) to select
a single individual out of the ten best. Selected individuals are then
remembered between generations, and their fitnesses artificially heightened,
with a decay over time.

For a phenome $p$, we concretely use the fitness function $F(P)$:
$$F(p) = \left|-(\left|V\right| + R_c + E_{\Delta} + E_{rest} + E_{posture} +
I)\right|$$
Where $|V|$ is the number of voxels, approximating the material cost of the
chair, $R_c$ is linear in the rotation of the chair after the rag doll
drop-experiment. $E_{\Delta}$ is an exponential term in the horizontal travel
distance of the rag doll (named Ethan), after the drop experiment. $E_{rest}$ is
an exponential term in the vertical distance from the rag doll's hips, to ca.
the middle of the chair, adjusted so that a chair carrying the rag doll above
the middle is not penalised. $E_{posture}$ is a linear term corresponding to how
far the rag doll's head is above his hips. Finally, $I$ is the term for
interactive evolution.

We tried many variations of the fitness function above; different scaling and
curves for the different terms, and expressing $F$ as products instead of a
sum, but none that gave significantly better chairs.

\subsection{Technologies}
The evolutionary experiments were performed using the
SharpNeat\cite{web:sharpneat} implementation of NEAT. Physics simulations were
done in Unity3D\cite{web:unity}. To integrate the two, we used Daniel Jallovs project, UnityNEAT\cite{web:unityneat}.

Jallov has done alot of the groundwork by porting a application of NEAT, SharpNeat\cite{web:sharpneat}, to the Unity3D game engine and setting up frameworks for describing evolutionary experiments\cite{web:unityneat}.

The rationale for using  a game engine to realize  this project, besides 
the fact that the majority of engines come with physics systems out of 
the box, is the opportunity to render the candidates which makes interactive 
evolution\cite{Woolley2014} easier to implement.

Much work went into using the mostly undocumented SharpNeat and UnityNEAT
projects, and for any consequent work, we would consider using a different
technology stack.
We tried many variations of the fitness function above, but none that
gave significantly better chairs.