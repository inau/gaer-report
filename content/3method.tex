\section{Evolving chairs using NEAT}
To solve a problem using genetic algorithms, we need an encoding
consisting of a genotype, phenotype, and a mapping from genotype to phenotype.

Encodings can generally be divided into two categories; direct and generative
(or indirect). Generative encodings have several properties that lend themselves
well to our problem domain. By using a genotype encoding that \textit{generates}
the phenotype, a single sub-structure in the genome can encode several similar
sub-structures in phenotype. This makes it significantly easier to evolve
individuals that have symmetry or repetition, but with variation --- such as
chairs! While the Hornby study \cite{paper:ev4}, we thought it would be
interesting to try a neuroevolutionary approach.

\subsection{Genome rep}
\subsection{Geometric algorithms}
\subsection{Fitnesses}
\subsection{Technologies (?)}

A solution using the Unity3D\cite{web:unity} game engine has been built. This has been done by building on a collegues work with evolutionary algorithms, particularly a project porting SharpNeat\cite{web:unityneat} to the Unity gameengine\cite{web:sharpneat}. While this is not directly applicable to our solution it supplied us with a framework for setting up evolutionary algorithms.

NEAT is short for neuro-evolution by augmenting topologies, its a technique which applies three key concepts to build complex neural networks, history, speciation and complexiation. First and foremost history is the 
    
\textbf{Unity} was an arbitrary choice. We could just as easily have chosen any other game engine which provided the same components - particularly physics and rendering, however Unity is used by thousands of people worldwide and has a lot of support provided by its active community which really works in favor of the engine.

