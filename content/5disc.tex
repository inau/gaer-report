\section{Discussion}
While our results are interesting in their own right, it is always prudent to 
consider what might have been done differently in order to obtain other results.
\todo[inline]{more/better introduction}

\subsection{Incremental Evolution}
An approach to be considered is that of incremental evolution. Incremental
evolution would have allowed us to focus on a subset of desired properties ,
such as the desired height, or stability. Once a population had been evolved
which fulfilled the desired properties, the fitness function could be modified
to add another property. 

Using this approach it is plausible that we would have gotten better results,
as the individuals would not have to consider e.g. Ethan's posture, before they
had evolved to an appropriate height.

\subsection{Simulation Performance}
Our experiments suffer from the computational cost of simulating physical
experiments. To compensate for the time it takes to run these experiments, we
run experiments for only a few hundredths of generations, or with small
population sizes. This does not allow NEAT enough time, in the case of few 
generations, or lebensraum, in the case of small populations, to properly work.

As mentioned in section \ref{sec:results} our population size was 50 for the 
experiments which only ran for a couple of hundred generations, and 10 when we 
wanted the experiments to run for more generations.
Given more time we would have liked to tweak the NEAT parameters, such as
complexification and speciation, and not only the evaluation function.

\subsection{Rag Doll - (Ethan)}
While the idea to use a rag doll as part of the evaluation to evaluate the 
"chairness" of the individuals is novel, we realised too late that the rag doll 
we used in our simulations was not as flexible as first assumed.

Ethan did not bend in his hips and knees, resulting in him not being able to 
actually sit down. This helps explain why the individuals which tried to 
minimize the $E_{posture}$ term would evolve into shapes as seen on
figures~\ref{fig:metro_stand}~\&~\ref{fig:chair_legs}, as these individuals 
show features which help Ethan stand upright.

It would be nice to do some more experiments where Ethan's flexibility more
closely resembled that of a human.

\subsection{Interactive Evolutionary Computation}
Our use of IEC differs from that used by Clune and
Lipson\cite{Clune:2011:EOG:2078245.2078246}, in that we only prompt the user
for input every few generations. In most of our experiments this was roughly
every tenth generations. It would be interesting to see what would happen if we
instead had prompted the user every generation. 

It would also be interesting to try and combine the above mentioned approach
with that of incremental evolution. First one could use only human evaluation
to produce a population of individuals with interesting shapes. Then apply the
static physical evaluation to evolve the desired physical properties, or vice
versa.
