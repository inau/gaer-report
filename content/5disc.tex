\section{Discussion}
\subsection{Unity Simulation Performance}
In general the simulations have performed quite horrible when having bigger populations.
Some ideas to improve performance is restricting how objects collide.
Either by creating layers matching the number of candidates in each generation
or make friction of drop object and terrain rather high.
Alternatively just remove any rigid body on the drop object when colliding with terrain.
The latter might require some additional tweaks to the test arena ie. some cubes that elevate the 
candidates such that an rag doll sitting in a chair with feet planted not collides with the surrounding terrain but only that elevate area.

\subsection{Voxel Mass on Candidates}
It has been quite troublesome to find a suitable weight for every voxel.
Seing the fitness depends on whether the candidate has moved or rotated it is quite important
to balance the mass of the drop test object in regards to the candidate objects mass, which varies alot based on morphology.
While the current mass used, 0.015 per voxel, seems to provide structures of roughly 1/4 of the 
drop objects mass, we would suggest to perform additional studies to determine the ideal value.
The current value was more or less chosen arbitrarly by trial-and-tweak until something which was 
not entirely unstable compared to our perception of how such physics objects should behave.

\subsection{Rag Doll - Ethan}
Looking at the results from the simulations it would have been very productive to be more
thorough when generating the rag doll for testing.
Especially when using the built-in tool in Unity, one should be very mindfull about which values are chosen for which extremities.
Incorrect values can force the extremities to move irradically or make arms pop out from the hips and similar.
