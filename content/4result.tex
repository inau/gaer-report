\section{Results} Several experiments were performed using the evaluation
function mentioned in section \ref{sec:evaluation}. 

Initially the experiments were carried out without the interactive term, $I$,
and without regard for the mass of the individuals produced by the EC. This
meant that the physical part of the evaluation, where Ethan was dropped on the
individuals, resulted in the terms, $E_\Delta$, $E_{rest}$, and $E_{posture}$
all being roughly equivalent between individuals. This happened because the 
higher mass of Ethan would always result in the individuals being rotated and 
Ethan dropping on the floor. This left only the $|V|$ and $R_c$ terms, and the 
EC ultimately optimized towards a minimal voxel count which would not rotate 
when Ethan was dropped, see figure~\ref{fig:flat_object}.

\begin{figure}[ht]
\centering
\includegraphics[scale=.5]{flat_chair}
\caption{The evolved individual, pink, is simply a flat structure, such that
it does not rotate when Ethan is dropped on it, and has minimal voxel
count.} \label{fig:flat_object} \end{figure}

Adding the interactive term, $I$, did not improve significantly on the above
results. While the selected individual would persist through several
generations, the other best performing individuals would still optimize towards
the aforementioned flat structure, see figure~\ref{fig:selection}.
\begin{figure}[ht]
	\centering
	\subfloat[]{
		\includegraphics[width=.9\columnwidth, trim={110 0 0 0}, clip]{selection}
	} \hfil
	\subfloat[]{
		\includegraphics[width=.9\columnwidth, trim={110 0 0 0}, clip]{selection2}
	}
	\caption{Example showing how the selected individual, encircled in (a),
	persists 8 generations later, encircled in (b).} \label{fig:selection}
\end{figure}

Further experiments were conducted, where the mass of the individuals was
adjusted according to their voxel count. In these experiments the EC ended up
producing some individuals which were not just flat individuals. Rather it produced
individuals as boxes on which Ethan could rest in a height such that the terms
$E_{rest}$ and $E_\Delta$ were minimized, see figure~\ref{fig:boxchair}.

\begin{figure}[ht]
	\centering
	\includegraphics[width=.9\columnwidth, trim={110 0 0 0}, clip]{box_chair}
	\caption{When considering the mass of the individuals, such that more
	compact individuals are more sturdy, the EC evolved a box upon which
	Ethan could rest.}
	\label{fig:boxchair}
\end{figure}

\begin{figure}[ht]
	\centering
	\subfloat[]{
		\includegraphics[width=.9\columnwidth]{basket}
	} \hfil
	\subfloat[]{
		\includegraphics[width=.9\columnwidth]{basket2}
	}
	\caption{Basket like shapes that might have functioned as chairs, had
	they had legs.}
	\label{fig:baskets}
\end{figure}

During experiments with spheres, we saw some promising results. The mat- and
tray-like shapes we had been seeing took on more basket-like shapes. We believe
that with more generations, they might have started to grow in height ---
perhaps with legs. Figure \ref{fig:baskets} depict some basket-like shapes.

\begin{figure}[ht]
	\centering
	\subfloat[]{
		\includegraphics[width=.9\columnwidth]{actual_chair}
		\label{fig:real_chair}
	} \hfil
	\subfloat[]{
		\includegraphics[height=70pt]{metro_stand}
		\label{fig:metro_stand}
	}
	\caption{Viable chairs. While the topmost chair is nothing more than
	viable, the bottom chair is a little surprising.}
	\label{fig:viable_chairs}
\end{figure}

\begin{figure}[ht]
	\centering
	\includegraphics[width=.9\columnwidth]{metro_stand_real}
	\caption{}
	\label{fig:metro_stand_real}
\end{figure}

We did see some viable chairs --- individuals that a person could use for seating.
Figure \ref{fig:viable_chairs} shows an example, not unlike a car seat. A more
surprising result can be seen in figure \ref{fig:metro_stand}. The chair bares
resemblance to the rests that can be found at metro platforms in Copenhagen, and
is an interesting diversion from common shape morphologies. One can be seen in
figure \ref{fig:metro_stand_real}. It is debatable whether an individual one leans
on can be considered a chair, but it suits the fitness function.

Some chair-like individuals were unfairly evaluated in simulation. Because the
simulation of "sitting" is simplified, some individuals were missed or even
toppled over in simulation. Figure \ref{fig:footstools} shows some chair-like
individuals that ended up as strange footstools.

\begin{figure}[ht]
	\centering
	\subfloat[]{
		\includegraphics[width=.9\columnwidth, trim={170 70 0 20}, clip]{chair_flipover} } \hfil
	\subfloat[]{
		\includegraphics[width=.9\columnwidth]{footstool}
	}
	\caption{The simulation is not entirely loyal to reality. Viable chairs
	might be deemed unfit because "sitting" is imperfectly simulated.}
	\label{fig:footstools}
\end{figure}
