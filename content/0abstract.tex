This project operates within the field of evolutionary computations.
The field of EC has prior examples of objects being generated by applying artificial neural networks or variations of such.
Examples of prior work are tables and space antennas using generative encoding and images 
generated by applying CPPNs with a interactive approach just to mention the ones which have 
influenced this project the most.

The novelty in our specific solution is the application of CPPN-NEAT to generate chairs, which is deemed a classical design issue.
This particular angle on EC ensures that we have repetition, symmetri and regularities in the candidates.
The results produced by the the network are quite diverse, the most frequently occuring 
shapes are trays, boxes and baskets.
The trays and baskets could potentially develop to something more similar to seating furniture known 
from the present, while the boxes already have some similarities to present day furniture.

While using CPPNs for creating chairs seems as it potentially could create some candidates suitable 
for sitting, we conclude that the application of CPPN-NEAT introduces some challenges.
Challenges such as defining a proper fitness function, making a efficient physics simulation or 
tweaking of NEAT parameters.
