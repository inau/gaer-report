The field of evolutionary computations has prior examples of objects being generated by applying artificial neural networks or variations of such.
Examples of prior work are tables and space antennas using generative encoding and images 
generated by applying CPPNs with an interactive approach.
Our specific solution applies CPPN-NEAT to generate chairs, which is deemed a classical design issue.
This particular angle on evolutionary computations ensures that we have repetition, symmetri and regularities in the candidates.
The results produced by the network are quite diverse, the most frequently occuring 
shapes are trays, boxes and baskets.
The trays and baskets could potentially develop to something more similar to seating furniture known 
from the present, while the boxes already have some similarities to present day furniture
While using CPPNs for creating chairs seems as it potentially could create some candidates suitable 
for sitting, we have not been able to produce any good chair candidates using CPPN-NEAT.
We have gotten structures which have similarities to the danish metro stands, structures which look like baskets and with further evolution might evolve legs and last but not least structures which resemble a 'L', which provide seating and back support.
