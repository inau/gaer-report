\section{Introduction}
Evolutionary Computation(EC) applies concepts from the domain of evolutionary
biology to problems in, among others, mathematics and computer science,
by simulating Darwinistic evolution and natural selection.

Genetic algorithms provide a Darwinian framework for local search algorithms.
Dividing the sub problems of optimization into encoding, selection,
reproduction, and evaluation. In this transformation from the real world of
biological evolution to EC, there is no inherent limitation binding EC to
software. It is possible to use these techniques on a computer, while solving
problems "in the real world". That is, we can take our experiments from
\emph{in silico} to \emph{in materio}\cite{paper:ev3}.

In this project, we explore the use of EC in morphologic furniture design.
Furniture design balances many objectives, such as material cost, production
complexity, novelty, aesthetic value, and storage space --- to point out a few.
EC lends itself well to problems that involve multiple objectives, and the idea
of automating creative/aesthetic work is an exhilarating idea in itself.

The goal is to produce viable chairs. Viable in the sense that a person should 
be able to sit in it. We hope to evolve chairs that, apart from being viable,
are unexpected or even novel results. EC can be an excellent means of novelty,
as one may, with care, eliminate a priori assumptions about the results, and let
the evolutionary process take control.

